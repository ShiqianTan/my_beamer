% 文档类型
\documentclass[16pt]{beamer}

% beamer主题
\usetheme{Madrid}
% \usecolortheme{beaver}

% 中英文环境
\usepackage{ctex}  %一个支持中文宏包,如果不用中文无法显示

% 插图
\usepackage{graphicx} %这个包提供\includegraphics命令来插入图片
\usepackage{subfigure} % 提供多列插图
% 数学环境
\usepackage{siunitx}
\usepackage{mathtools} % 引入数学:=定义符号
\usepackage{physics} % 使用微分符号\dd
% 圆圈中带数字
\usepackage{tikz}
\newcommand*{\circled}[1]{\lower.7ex\hbox{\tikz\draw (0pt, 0pt)%
        circle (.5em) node {\makebox[1em][c]{\small #1}};}}

% 引用
\usepackage{hyperref}
\hypersetup{
    colorlinks=true,
    linkcolor=blue, % white导致中文beamer看不出目录
    filecolor=blue,      
    urlcolor=blue,
    citecolor=cyan,
    pdfauthor=ShiqianTan
}

%-----行距设置-----
%1.2代表的是1.5倍行距;设置为2倍行距应打入1.667
\linespread{1.0}

% 设置日期格式
\renewcommand{\today}{\number\year 年\number\month 月\number\day 日}

% 首页
\title[第二次习题课]{量子化学与原理}
\subtitle{第二次习题课}
\author[Shiqian Tan]{谭诗乾}
\institute[Fudan University]{复旦大学化学系}
\date{\today} %时间(默认也会显示)
% 使用logo
% \logo{\includegraphics[height=0.50cm]{./pictures/FudanLogo.jpg}}

% 正文
\begin{document}
    
    % 首页
    \frame{\titlepage}
    
    % 总目录页
    \begin{frame}
        \frametitle{目录}
        \tableofcontents[hideallsubsections]
    \end{frame}
    
    % 每个section自动添加目录页
    \AtBeginSection[]{
        \frame{\frametitle{目录}\tableofcontents[
            sectionstyle=show/shaded,
            subsectionstyle=show/show/shaded]}
    }
    
    % section自动目录,不在handout中生成
    % \AtBeginSection[]{
    % \begin{frame}<beamer>
    %     \frametitle{目录}\tableofcontents[
    %     sectionstyle=show/shaded,
    %     subsectionstyle=show/show/shaded]
    % \end{frame}
    % }
    
    \section{目录1}
    \subsection{子目录1}
    
    \section{目录2}
    \subsection{子目录2}
    
    
    \begin{frame}
        \frametitle{}
        \Huge
        \begin{center}
            感谢聆听!
        \end{center}
    \end{frame}
    
\end{document}
